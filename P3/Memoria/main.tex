\input{estilo.tex}
\usepackage{textcomp}
\usepackage{hyperref}

%----------------------------------------------------------------------------------------
%	DATOS
%----------------------------------------------------------------------------------------

\newcommand{\myName}{Francisco Javier Bolívar Lupiáñez}
\newcommand{\myMail}{fblupi@correo.ugr.es}
\newcommand{\myDNI}{75926571-Y}
\newcommand{\myDegree}{Máster en Ingeniería Informática}
\newcommand{\myFaculty}{E. T. S. de Ingenierías Informática y de Telecomunicación}
\newcommand{\myDepartment}{Ciencias de la Computación e Inteligencia Artificial}
\newcommand{\myUniversity}{\protect{Universidad de Granada}}
\newcommand{\myLocation}{Granada}
\newcommand{\myTime}{\today}
\newcommand{\myTitle}{Práctica 3}
\newcommand{\mySubtitle}{Desarrollo de un Sistema de Recuperación de Información con Lucene}
\newcommand{\mySubject}{Gestión de Información en la Web}
\newcommand{\myYear}{2016-2017}

%----------------------------------------------------------------------------------------
%	PORTADA
%----------------------------------------------------------------------------------------


\title{	
	\normalfont \normalsize 
	\textsc{\textbf{\mySubject \space (\myYear)} \\ \myDepartment} \\[20pt] % Your university, school and/or department name(s)
	\textsc{\myDegree \\[10pt] \myFaculty \\ \myUniversity} \\[25pt]
	\horrule{0.5pt} \\[0.4cm] % Thin top horizontal rule
	\huge \myTitle: \mySubtitle \\ % The assignment title
	\horrule{2pt} \\[0.5cm] % Thick bottom horizontal rule
	\normalfont \normalsize
}

\author{
	\myName \\
	\small \texttt{\myMail} \\
	\small \myDNI \\
}

\date{\myTime} % Incluye la fecha actual
%----------------------------------------------------------------------------------------
%	INDICE
%----------------------------------------------------------------------------------------

\begin{document}
	
\definecolor{light-gray}{gray}{0.95}
	
\lstset {
	basicstyle=\scriptsize,
	frame=single,
	backgroundcolor=\color{grey}
}
	
\setcounter{page}{0}

\maketitle % Muestra el Título
\thispagestyle{empty}

\newpage %inserta un salto de página

\tableofcontents % para generar el índice de contenidos

\newpage %inserta un salto de página

\listoffigures

%\listoftables

\newpage

%----------------------------------------------------------------------------------------
%	DOCUMENTO
%----------------------------------------------------------------------------------------

\section{Introducción}
\label{sec:intro}

Los objetivos de esta práctica son:

\begin{itemize}
	\item Conocer las partes principales que tiene un sistema de recuperación de información y qué funcionalidad tiene cada una.
	\item Implementar un sistema de recuperación de información.
	\item Emplear la biblioteca \textbf{Lucene} para facilitar la implementación.
\end{itemize}

Para el desarrollo se ha utilizado \textbf{Java}, pues es el lenguaje de programación para el que Lucene cuenta con más documentación, además de por la experiencia previa con éste.
\\ \\
Se ha utilizado \textbf{IntelliJ IDEA Community Edition 2017.1} como IDE\footnote{\textit{Integrated Development Environment} (Entorno de desarrollo integrado)} y \textbf{Swing} para crear la GUI\footnote{\textit{Graphic User Interface} (Interfaz Gráfica de Usuario)}.
\\ \\
Se implementarán dos programas:

\begin{itemize}
	\item \textbf{Indexador}: Recibe como argumentos la ruta de la colección documental a indexar, el fichero de palabras vacías a emplear y la ruta donde se alojarán los índices. Se ejecutará desde la línea de comandos sin ninguna GUI y llevará a cabo la indexación, creando los índices oportunos y ficheros auxiliares necesarios para la recuperación.
	\item \textbf{Motor de búsqueda}: A diferencia del indexador, éste sí contará con una GUI desde la que se podrá elegir el directorio donde están alojados los índices y permitirá realizar una búsqueda sobre estos.
\end{itemize}

\section{Desarrollo}
\label{sec:desarrollo}

Se ha desarrollado un indexador y motor de búsqueda básicos sin ningún elemento innovador.
\\ \\
Los códigos son muy parecidos a los proporcionados, solo que utilizando un \textit{SpanishAnalyzer} en lugar del \textit{Analyzer} estándar.
\\ \\
Se encontraron distintos problemas a la hora de leer los ficheros pues se leyeron con el DOM de Java y no tenían el formato adecuado para ello por lo que se tuvo que englobar todo el contenido de cada fichero entre dos etiquetas \texttt{<SGML>...</SGML>} y se tuvieron que cambiar algunas secuencias de caracteres que causaban fallos en la lectura.
\\ \\
El código está disponible en \href{https://github.com/fblupi/master_informatica-GIW/tree/master/P3/InformationRetrievalSystem}{GitHub}.

\section{Manual de usuario}
\label{sec:manual}

\subsection{Indexador}

Para indexar tan solo hay que ejecutar el programa (\textit{Indexer}). Los argumentos se encuentran dentro del código (como rutas relativas), por lo que para cambiar las rutas habría que recompilarlo.

\subsection{Motor de Búsqueda}

Para lanzar el motor de búsqueda hay que lanzar el programa \textit{GUI} que ejecutará la interfaz de usuario con el motor de búsqueda (Figura \ref{fig:init}).
\\ \\
Utiliza la misma ruta para las palabras vacías que el indexador, y al igual que éste, para cambiar la ruta habría que recompilar.

\begin{figure}[H]
	\centering
	\includegraphics[width=12cm]{img/init}
	\caption{Vista de la GUI recién iniciado el programa}
	\label{fig:init}
\end{figure}

Una vez iniciado el programa solo nos permitirá importar los índices, para ello hay que pulsar en el botón y seleccionar el directorio que los contiene (Figura \ref{fig:index-folder-selection}).

\begin{figure}[H]
	\centering
	\includegraphics[width=8cm]{img/index-folder-selection}
	\caption{Selección del directorio con los índices}
	\label{fig:index-folder-selection}
\end{figure}

Ya con los índices cargados se podrán realizar búsquedas. En la izquierda aparecerán los títulos de todos los resultados encontrados, a la derecha el texto completo cuando se selecciona cualquiera de ellos y abajo a la izquierda el número de resultados y el tiempo que ha tardado en encontrarlos (Figura \ref{fig:search}).

\begin{figure}[H]
	\centering
	\includegraphics[width=12cm]{img/search}
	\caption{Búsqueda del término perro}
	\label{fig:search}
\end{figure}

%----------------------------------------------------------------------------------------
%	REFERENCIAS
%----------------------------------------------------------------------------------------

%\newpage

%\bibliography{referencias} %archivo referencias.bib que contiene las entradas 
%\bibliographystyle{plain} % hay varias formas de citar

\end{document}