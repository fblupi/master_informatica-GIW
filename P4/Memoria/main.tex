\input{estilo.tex}
\usepackage{textcomp}
\usepackage{hyperref}

%----------------------------------------------------------------------------------------
%	DATOS
%----------------------------------------------------------------------------------------

\newcommand{\myName}{Francisco Javier Bolívar Lupiáñez}
\newcommand{\myMail}{fblupi@correo.ugr.es}
\newcommand{\myDNI}{75926571-Y}
\newcommand{\myDegree}{Máster en Ingeniería Informática}
\newcommand{\myFaculty}{E. T. S. de Ingenierías Informática y de Telecomunicación}
\newcommand{\myDepartment}{Ciencias de la Computación e Inteligencia Artificial}
\newcommand{\myUniversity}{\protect{Universidad de Granada}}
\newcommand{\myLocation}{Granada}
\newcommand{\myTime}{\today}
\newcommand{\myTitle}{Práctica 4}
\newcommand{\mySubtitle}{Desarrollo de un Sistema de Recomendación basado en Filtrado Colaborativo}
\newcommand{\mySubject}{Gestión de Información en la Web}
\newcommand{\myYear}{2016-2017}

%----------------------------------------------------------------------------------------
%	PORTADA
%----------------------------------------------------------------------------------------


\title{	
	\normalfont \normalsize 
	\textsc{\textbf{\mySubject \space (\myYear)} \\ \myDepartment} \\[20pt] % Your university, school and/or department name(s)
	\textsc{\myDegree \\[10pt] \myFaculty \\ \myUniversity} \\[25pt]
	\horrule{0.5pt} \\[0.4cm] % Thin top horizontal rule
	\huge \myTitle: \mySubtitle \\ % The assignment title
	\horrule{2pt} \\[0.5cm] % Thick bottom horizontal rule
	\normalfont \normalsize
}

\author{
	\myName \\
	\small \texttt{\myMail} \\
	\small \myDNI \\
}

\date{\myTime} % Incluye la fecha actual
%----------------------------------------------------------------------------------------
%	INDICE
%----------------------------------------------------------------------------------------

\begin{document}
	
\definecolor{light-gray}{gray}{0.95}
	
\lstset {
	basicstyle=\scriptsize,
	frame=single,
	backgroundcolor=\color{grey},
	breaklines=true
}
	
\setcounter{page}{0}

\maketitle % Muestra el Título
\thispagestyle{empty}

\newpage %inserta un salto de página

\tableofcontents % para generar el índice de contenidos

\newpage %inserta un salto de página

%\listoffigures

%\listoftables

\newpage

%----------------------------------------------------------------------------------------
%	DOCUMENTO
%----------------------------------------------------------------------------------------

\section{Introducción}
\label{sec:intro}

Los objetivos de esta práctica son:

\begin{itemize}
	\item Entender el proceso de recomendación basado en filtrado colaborativo.
	\item Ser capaz de plasmarlo en un programa de ordenador.
\end{itemize}

Para el desarrollo se ha utilizado \textbf{Java} por la experiencia previa con éste.
\\ \\
Se ha utilizado \textbf{IntelliJ IDEA Community Edition 2017.1} como IDE\footnote{\textit{Integrated Development Environment} (Entorno de desarrollo integrado)}.
\\ \\
Se implementarán un programa:

\begin{itemize}
	\item \textbf{Recommender}: El cual obtiene las valoraciones de películas de un usuario y obtiene su vecindario a partir de la base de datos \textit{Movie Lens} utilizando la correlación de Pearson para, a continuación, obtener películas recomendadas.
\end{itemize}

\section{Desarrollo}
\label{sec:desarrollo}

En primer lugar se han leído tanto la información de películas como de valoraciones de los archivos \texttt{u.item} y \texttt{u.data} respectivamente de la base de datos de \textit{Movie Lens}. 

\begin{itemize}
	\item Las películas se han almacenado en un diccionario donde la clave es el id de la película y el valor el nombre.
	\item Las recomendaciones se han almacenado en un diccionario donde la clave es el id del usuario y la clave un diccionario en el que la clave es el id de la película y el valor la valoración del usuario.
	\item Adicionalmente se ha creado otro diccionario en el que la clave es el id del usuario y el valor la valoración media que da a las películas. Ya que este datos se utiliza mucho.
\end{itemize}

Lo siguiente que he realizado ha sido obtener el vecindario. Para ello, además de las valoraciones extraídas de la base de datos se necesitan valoraciones del usuario y el número de vecinos más cercanos que se quiere.
\\ \\
Para obtener las puntuaciones se puede hacer introduciendo datos a mano o aleatoriamente cambiando un \textit{flag} en el \texttt{main}.
\\ \\
Una vez se tienen estos datos se pasa a calcular la similitud de todos los usuarios extraídos de la base de datos de \textit{Movie Lens}. Para ello se ha utilizado la correlación de Pearson:

\[ sim(u,v) = \frac{\sum(r(u,i)-\bar{r}(u))(r(v,i)-\bar{r}(v))}{\sqrt{\sum(r(u,i)-\bar{r}(u))^2}\sqrt{\sum(r(v,i)-\bar{r}(v))^2}} \]

Donde $ r(u,i) $ es la valoración del usuario \textit{u} a la película \textit{i} y $ \bar{r}(u) $ la valoración media del usuario \textit{u}.
\\ \\
Una vez obtenidos todos los usuarios se han ordenado (\url{http://stackoverflow.com/questions/109383/sort-a-mapkey-value-by-values-java}) y se han cogido los \textit{k} primeros.
\\ \\
Con estos se puede pasar a calcular la recomendación para el resto de películas:

\[ \hat{r}(u,i) = \bar{r}(u) + \frac{sim(u,v)(r(v,i)-\bar{r}(v))}{\sum|sim(u,v)|} \]

Donde $ r(v,i) $ es la valoración del usuario \textit{v} a la película \textit{i}, $ \bar{r}(u) $ la valoración media del usuario \textit{u} y $ sim(u,v) $ la similitud entre los usuarios \textit{u} y \textit{v} calculada anteriormente.
\\ \\
Por último se han seleccionado las \textit{n} películas con más valoración siempre y cuando esta valoración supere el umbral de 4 estrellas.
\\ \\
El código está disponible en \href{https://github.com/fblupi/master_informatica-GIW/tree/master/P3/InformationRetrievalSystem}{GitHub}.

\section{Manual de usuario}
\label{sec:manual}

El programa se ejecutará en línea de órdenes.
\\ \\
Tan solo hay que ejecutar el programa e introducir valoraciones, el programa devolverá los resultados. Es necesario tener los datos \texttt{u.data} y \texttt{u.item} en el directorio \texttt{data/ml-data} para que el programa los pueda leer.
\\ \\
Para cambiar cualquiera de los \textit{flags} (número de valoraciones de entrada, número de vecinos más cercanos, número de recomendaciones, puntuación mínima para recomendar y valoraciones de entrada aleatorias o no) hay que cambiar el código y recompilar.

\subsection{Ejemplo de salida}

\begin{lstlisting}
**********************************************
Obtaining user data: 
**********************************************
Movie: Boys on the Side (1995)
Enter your rating (1-5):
1
Movie: Jungle Book, The (1994)
Enter your rating (1-5):
2
Movie: Man Without a Face, The (1993)
Enter your rating (1-5):
3
Movie: First Kid (1996)
Enter your rating (1-5):
2
Movie: Sphere (1998)
Enter your rating (1-5):
1
Movie: Cliffhanger (1993)
Enter your rating (1-5):
5
Movie: Ghosts of Mississippi (1996)
Enter your rating (1-5):
3
Movie: Glass Shield, The (1994)
Enter your rating (1-5):
3
Movie: Death and the Maiden (1994)
Enter your rating (1-5):
1
Movie: Naked in New York (1994)
Enter your rating (1-5):
4
Movie: Barcelona (1994)
Enter your rating (1-5):
1
Movie: Lawnmower Man, The (1992)
Enter your rating (1-5):
3
Movie: Pete's Dragon (1977)
Enter your rating (1-5):
3
Movie: Wings of Courage (1995)
Enter your rating (1-5):
4
Movie: Terminator, The (1984)
Enter your rating (1-5):
1
Movie: Smilla's Sense of Snow (1997)
Enter your rating (1-5):
4
Movie: Executive Decision (1996)
Enter your rating (1-5):
5
Movie: Only You (1994)
Enter your rating (1-5):
2
Movie: Fish Called Wanda, A (1988)
Enter your rating (1-5):
4
Movie: Thin Line Between Love and Hate, A (1996)
Enter your rating (1-5):
3
**********************************************
Recommendations: 
**********************************************
Movie: Cop Land (1997), Rating: 4.544520547945206
Movie: Underground (1995), Rating: 4.470496894409938
Movie: Faust (1994), Rating: 4.470496894409938
Movie: 8 1/2 (1963), Rating: 4.470496894409938
Movie: City of Lost Children, The (1995), Rating: 4.470496894409938
Movie: Nosferatu (Nosferatu, eine Symphonie des Grauens) (1922), Rating: 4.470496894409938
Movie: Close Shave, A (1995), Rating: 4.470496894409938
Movie: This Is Spinal Tap (1984), Rating: 4.470496894409938
Movie: Grand Day Out, A (1992), Rating: 4.470496894409938
Movie: Delicatessen (1991), Rating: 4.470496894409938
Movie: Cinema Paradiso (1988), Rating: 4.470496894409938
Movie: Wrong Trousers, The (1993), Rating: 4.470496894409938
Movie: Sleeper (1973), Rating: 4.470496894409938
Movie: Three Colors: White (1994), Rating: 4.470496894409938
Movie: Three Colors: Blue (1993), Rating: 4.470496894409938
Movie: Three Colors: Red (1994), Rating: 4.470496894409938
Movie: Postino, Il (1994), Rating: 4.470496894409938
Movie: Mrs. Brown (Her Majesty, Mrs. Brown) (1997), Rating: 4.230593607305936
Movie: Anastasia (1997), Rating: 4.228260869565217
Movie: Manchurian Candidate, The (1962), Rating: 4.1535087719298245
\end{lstlisting}

%----------------------------------------------------------------------------------------
%	REFERENCIAS
%----------------------------------------------------------------------------------------

%\newpage

%\bibliography{referencias} %archivo referencias.bib que contiene las entradas 
%\bibliographystyle{plain} % hay varias formas de citar

\end{document}